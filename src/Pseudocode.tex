\documentclass{article}
\usepackage{amsmath}
\usepackage{algorithm}
\usepackage{algpseudocode}
\usepackage{fullpage}

\begin{document}

\section*{Pseudocode}

\begin{algorithm}
\caption{Generating Pseudopeople Data}
\begin{algorithmic}[1]
\State \textbf{BEGIN}
\State \hspace{1cm} \texttt{PRINT "Generating Pseudopeople Data"}
\State \hspace{1cm} \texttt{SET src\_1 TO '[Path to Data]'} 
\State \hspace{1cm} \texttt{SET src\_2 TO '[Path to Data 2]'} 
\State \hspace{1cm} \texttt{READ CSV file from src\_1 with headers and tab separator, STORE in df\_1}
\State \hspace{1cm} \texttt{READ CSV file from src\_2 with headers and tab separator, STORE in df\_2}
\State \textbf{END}
\end{algorithmic}
\end{algorithm}

\begin{algorithm}
\caption{Sorting Data}
\begin{algorithmic}[1]
\State \textbf{BEGIN}
\State \hspace{1cm} \texttt{SORT df\_1 by columns "last\_name", "first\_name", "middle\_initial", "age", "street\_name" in ascending order}
\State \hspace{1cm} \texttt{STORE sorted result in df\_1\_sorted}
\State \hspace{1cm} \texttt{SORT df\_2 by columns "last\_name", "first\_name", "middle\_initial", "age", "street\_name" in ascending order}
\State \hspace{1cm} \texttt{STORE sorted result in df\_2\_sorted}
\State \textbf{END}
\end{algorithmic}
\end{algorithm}

\begin{algorithm}
\caption{Deduplication and Counting Rows}
\begin{algorithmic}[1]
\State \textbf{BEGIN}
\State \hspace{1cm} \texttt{REMOVE duplicates from df\_sorted based on all columns except the first column (from index 1 onwards)}
\State \hspace{1cm} \texttt{STORE the result in df\_deduplicated}
\State \hspace{1cm} \texttt{COUNT the number of rows in df\_deduplicated}
\State \hspace{1cm} \texttt{STORE the count result}
\State \textbf{END}
\end{algorithmic}
\end{algorithm}

\begin{algorithm}
\caption{K-mer Generation}
\begin{algorithmic}[1]
\State \textbf{BEGIN}
\State \hspace{1cm} \texttt{DEFINE k as 3}
\State \hspace{1cm} \texttt{DEFINE function generate\_k\_mer with input str\_d:}
\State \hspace{2cm} \texttt{IF length of str\_d is less than or equal to k:}
\State \hspace{3cm} \texttt{RETURN a list containing str\_d}
\State \hspace{2cm} \texttt{ELSE:}
\State \hspace{3cm} \texttt{RETURN a list of substrings of length k starting from each position in str\_d}
\State \hspace{1cm} \texttt{DEFINE kmer\_udf as a user-defined function (UDF) to apply generate\_k\_mer to each input sequence}
\State \hspace{1cm} \texttt{APPLY kmer\_udf to the "last\_name" column of df\_deduplicated}
\State \hspace{1cm} \texttt{STORE the result in df\_2\_new with a new column "kmers"}
\State \hspace{1cm} \texttt{ADD a new "id" column to df\_2\_new using row\_number() based on an order of monotonically increasing ID}
\State \hspace{1cm} \texttt{STORE the result in df\_with\_kmers}
\State \hspace{1cm} \texttt{DISPLAY the first 10 rows of df\_with\_kmers}
\State \textbf{END}
\end{algorithmic}
\end{algorithm}

\begin{algorithm}
\caption{Blocking and Grouping K-mers}
\begin{algorithmic}[1]
\State \textbf{BEGIN}
\State \hspace{1cm} \texttt{DEFINE function do\_blocking with input df:}
\State \hspace{2cm} \texttt{EXPLODE the "kmers" column into individual (kmer, index) pairs}
\State \hspace{2cm} \texttt{STORE the result in exploded\_df with "kmer" and "index" columns}
\State \hspace{2cm} \texttt{CONVERT the "kmer" column to lowercase for consistency}
\State \hspace{2cm} \texttt{GROUP exploded\_df by "kmer" and collect the indices in a list for each unique kmer}
\State \hspace{2cm} \texttt{STORE the result in blocked\_df}
\State \hspace{1cm} \texttt{APPLY do\_blocking function to df\_with\_kmers}
\State \hspace{1cm} \texttt{STORE the result in blocked\_df}
\State \hspace{1cm} \texttt{DISPLAY the first 5 rows of blocked\_df}
\State \textbf{END}
\end{algorithmic}
\end{algorithm}

\begin{algorithm}
\caption{Creating Pairs from Blocking Results}
\begin{algorithmic}[1]
\State \textbf{BEGIN}
\State \hspace{1cm} \texttt{DEFINE function all\_pairs with input df:}
\State \hspace{2cm} \texttt{CREATE a cross join between df and df, renaming the "list\_indices" column to "id1" in the first DataFrame}
\State \hspace{2cm} \texttt{AND renaming the "list\_indices" column to "id2" in the second DataFrame}
\State \hspace{1cm} \texttt{STORE the result of the cross join in df\_cross}
\State \hspace{1cm} \texttt{APPLY all\_pairs function to df}
\State \hspace{1cm} \texttt{STORE the result in df\_cross}
\State \textbf{END}
\end{algorithmic}
\end{algorithm}

\end{document}
